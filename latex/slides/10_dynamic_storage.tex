%% Nothing to modify here.
%% make sure to include this before anything else

\documentclass[10pt]{beamer}
\usetheme{Szeged}

% packages
\usepackage{color}
\usepackage{listings}

% color definitions
\definecolor{mygreen}{rgb}{0,0.6,0}
\definecolor{mygray}{rgb}{0.5,0.5,0.5}
\definecolor{mymauve}{rgb}{0.58,0,0.82}

% re-format the title frame page
\makeatletter
\def\supertitle#1{\gdef\@supertitle{#1}}%
\setbeamertemplate{title page}
{
  \vbox{}
  \vfill
  \begin{centering}
  \begin{beamercolorbox}[sep=8pt,center]{title}
      \usebeamerfont{supertitle}\@supertitle
   \end{beamercolorbox}
    \begin{beamercolorbox}[sep=8pt,center]{title}
      \usebeamerfont{title}\inserttitle\par%
      \ifx\insertsubtitle\@empty%
      \else%
        \vskip0.25em%
        {\usebeamerfont{subtitle}\usebeamercolor[fg]{subtitle}\insertsubtitle\par}%
      \fi%     
    \end{beamercolorbox}%
    \vskip1em\par
    \begin{beamercolorbox}[sep=8pt,center]{author}
      \usebeamerfont{author}\insertauthor
    \end{beamercolorbox}
    \begin{beamercolorbox}[sep=8pt,center]{institute}
      \usebeamerfont{institute}\insertinstitute
    \end{beamercolorbox}
    \begin{beamercolorbox}[sep=8pt,center]{date}
      \usebeamerfont{date}\insertdate
    \end{beamercolorbox}\vskip0.5em
    {\usebeamercolor[fg]{titlegraphic}\inserttitlegraphic\par}
  \end{centering}
  \vfill
}
\makeatother

% insert frame number
\expandafter\def\expandafter\insertshorttitle\expandafter{%
      \insertshorttitle\hfill%
\insertframenumber\,/\,\inserttotalframenumber}

% preset-listing options
\lstset{
  backgroundcolor=\color{white},   
  % choose the background color; 
  % you must add \usepackage{color} or \usepackage{xcolor}
  basicstyle=\footnotesize,        
  % the size of the fonts that are used for the code
  breakatwhitespace=false,         
  % sets if automatic breaks should only happen at whitespace
  breaklines=true,                 % sets automatic line breaking
  captionpos=b,                    % sets the caption-position to bottom
  commentstyle=\color{mygreen},    % comment style
  % deletekeywords={...},            
  % if you want to delete keywords from the given language
  extendedchars=true,              
  % lets you use non-ASCII characters; 
  % for 8-bits encodings only, does not work with UTF-8
  frame=single,                    % adds a frame around the code
  keepspaces=true,                 
  % keeps spaces in text, 
  % useful for keeping indentation of code 
  % (possibly needs columns=flexible)
  keywordstyle=\color{blue},       % keyword style
  % morekeywords={*,...},            
  % if you want to add more keywords to the set
  numbers=left,                    
  % where to put the line-numbers; possible values are (none, left, right)
  numbersep=5pt,                   
  % how far the line-numbers are from the code
  numberstyle=\tiny\color{mygray}, 
  % the style that is used for the line-numbers
  rulecolor=\color{black},         
  % if not set, the frame-color may be changed on line-breaks 
  % within not-black text (e.g. comments (green here))
  stepnumber=1,                    
  % the step between two line-numbers. 
  % If it's 1, each line will be numbered
  stringstyle=\color{mymauve},     % string literal style
  tabsize=4,                       % sets default tabsize to 4 spaces
  title=\lstname                   
  % show the filename of files included with \lstinputlisting; 
  % also try caption instead of title
}

% macro for code inclusion
\newcommand{\includecode}[2][c]{
	\lstinputlisting[caption=#2, style=custom#1]{#2}
}	% nothing to do here
\usepackage[english]{babel}

\usepackage[utf8]{inputenc}

\newcommand{\course}{
	C introduction
}

\author{
	Richard Mörbitz,
	Manuel Thieme
}

\lstset{
	language = C,
	showspaces = false,
	showtabs = false,
	showstringspaces = false,
	tabsize = 4,
	escapechar = @
} % TODO modify this if you have not already done so

% meta-information
\newcommand{\topic}{
	Dynamic memory management
}
\usepackage{tikz}
\usetikzlibrary{arrows}
\usetikzlibrary{decorations.pathmorphing}
\tikzset{arrow/.style={-latex, shorten >=.5ex, shorten <=.5ex}}

% nothing to do here
\title{\topic}
\supertitle{\course}
\date{}

% the actual document
\begin{document}

\maketitle

\begin{frame}{Contents}
	\tableofcontents
\end{frame}

\section{Motivation}
\subsection{}
\begin{frame}{Runtime conditions}
	Static C arrays are great if you already know at \textit{compile time} how many elements you will need later at \textit{runtime}. \\ \ \\
	Seriously, how often is this the case? \\ \ \\
	\begin{uncoverenv}<2->
		\textit{Never}. Unless your program does not take \textit{any} user input, you cannot determine how much data you will have to store. \\ \ \\
		This is where dynamic memory management comes into play.
	\end{uncoverenv}
\end{frame}
\begin{frame}{Explicit allocation}
	All the variables and arrays you have used so far were placed in memory automatically. 
	In dynamic memory management, you have to \textit{allocate} parts memory to identifiers on your own. \\ \ \\
	There are four functions in the standard library that do almost all the work for you:
	\begin{itemize}
		\item \textit{malloc()}: Allocate a block of memory
		\item \textit{calloc()}: Allocate a block of memory and initialize it
		\item \textit{realloc()}: Alter the size of a block of memory
		\item \textit{free()}: Release a block of memory
	\end{itemize}
	They are declared in \textit{stdlib.h}.
\end{frame}
\section{Memory allocation}
\subsection{}
\begin{frame}[fragile]{A closer look at memory}
	\begin{tikzpicture}[y=2cm, font=\footnotesize]
		\node[above, font=\small] at (1.5,0) {Stack};
		
		\draw (0,0) -- (0,-2);
		\draw (0,0) -- (3,0);
		\draw (3,0) -- (3,-2);
		\draw[line join=round, decoration={zigzag,segment length=4, amplitude=1}, decorate] (0,-2) -- (3,-2);
		
		\draw[->] (-.25,0) -- (-.25,-1.9);
		
		\draw[dashed] (0,-.25) -- (3,-.25);
		\draw[dashed] (0,-.5) -- (3,-.5);
		\draw[dashed] (0,-.75) -- (3,-.75);
		\draw[dashed] (0,-1) -- (3,-1);
		\draw[dashed] (0,-1.25) -- (3,-1.25);
		\draw[dashed] (0,-1.5) -- (3,-1.5);
		\draw[dashed] (0,-1.75) -- (3,-1.75);
		
		\node[below, font=\small] at (6.5,-2) {Heap};		
		
		\draw (5,0) -- (5,-2);
		\draw[line join=round, decoration={zigzag,segment length=4, amplitude=1}, decorate] (5,0) -- (8,0);
		\draw (8,0) -- (8,-2);
		\draw (5,-2) -- (8,-2);
		
		\draw[<-] (8.25,-.1) -- (8.25,-2);
		
		\draw[dashed] (5,-.25) -- (8,-.25);
		\draw[dashed] (5,-.5) -- (8,-.5);
		\draw[dashed] (5,-.75) -- (8,-.75);
		\draw[dashed] (5,-1) -- (8,-1);
		\draw[dashed] (5,-1.25) -- (8,-1.25);
		\draw[dashed] (5,-1.5) -- (8,-1.5);
		\draw[dashed] (5,-1.75) -- (8,-1.75);
		
		\draw[thick, dashed, orange] (0,0) -- (3,0);
		\draw[thick, dashed, orange] (0,0) -- (0,-.25);
		\draw[thick, dashed, orange] (0,-.25) -- (3,-.25);
		\draw[thick, dashed, orange] (3,0) -- (3,-.25);
		\node[orange,below=.15em] at (1.5,0){int};

		\draw[thick, dashed, orange] (0,-.25) -- (3,-.25);
		\draw[thick, dashed, orange] (0,-.25) -- (0,-.5);
		\draw[thick, dashed, orange] (0,-.5) -- (3,-.5);
		\draw[thick, dashed, orange] (3,-.25) -- (3,-.5);
		\node[orange,below=.15em] at (1.5,-.25){int};
		
		\begin{uncoverenv}<3->
			\draw[thick, dashed, orange] (5,-1.75) -- (8,-1.75);
			\draw[thick, dashed, orange] (5,-1.75) -- (5,-2);
			\draw[thick, dashed, orange] (5,-2) -- (8,-2);
			\draw[thick, dashed, orange] (8,-1.75) -- (8,-2);
			\node[orange,below=.15em] at (6.5,-1.75){int};
		\end{uncoverenv}
		
		\begin{uncoverenv}<4->
			\draw[thick, dashed, teal] (0,-.5) -- (3,-.5);
			\draw[thick, dashed, teal] (0,-.5) -- (0,-1);
			\draw[thick, dashed, teal] (0,-1) -- (3,-1);
			\draw[thick, dashed, teal] (3,-.5) -- (3,-1);
			\node[teal,below=.15em] at (1.5,-.5){int*};
			
			\draw (3,-.75) edge[out=0,in=180,->,shorten >=.5ex, shorten <=.5ex] (5,-1.875);
		\end{uncoverenv}
		
		\begin{uncoverenv}<5->
			\draw[thick, dashed, purple] (5,-1.25) -- (8,-1.25);
			\draw[thick, dashed, purple] (5,-1.25) -- (5,-1.75);
			\draw[thick, dashed, purple] (5,-1.75) -- (8,-1.75);
			\draw[thick, dashed, purple] (8,-1.25) -- (8,-1.75);
			\node[purple, below=.15em] at (6.5,-1.25){long};
			
			\draw[thick, dashed, teal] (0,-1) -- (3,-1);
			\draw[thick, dashed, teal] (0,-1) -- (0,-1.5);
			\draw[thick, dashed, teal] (0,-1.5) -- (3,-1.5);
			\draw[thick, dashed, teal] (3,-1) -- (3,-1.5);
			\node[teal,below=.15em] at (1.5,-1){long*};
			
			\draw (3,-1.25) edge[out=0,in=180,->,shorten >=.5ex, shorten <=.5ex] (5,-1.5);
		\end{uncoverenv}
		
		% Leave this here for historic reasons:
		%\begin{uncoverenv}<6->
		%	\draw[thick, dashed, purple] (5,-.75) -- (8,-.75);
		%	\draw[thick, dashed, purple] (5,-.75) -- (5,-1.25);
		%	\draw[thick, dashed, purple] (5,-1.25) -- (8,-1.25);
		%	\draw[thick, dashed, purple] (8,-.75) -- (8,-1.25);
		%	\node[purple, below=.15em] at (6.5,-.75){long};
		%	
		%	\draw[thick, dashed, teal] (0,-1.5) -- (3,-1.5);
		%	\draw[thick, dashed, teal] (0,-1.5) -- (0,-2);
		%	\draw[thick, dashed, teal] (0,-2) -- (3,-2);
		%	\draw[thick, dashed, teal] (3,-1.5) -- (3,-2);
		%	\node[teal,below=.15em] at (1.5,-1.5){long*};
		%	
		%	\draw (3,-1.75) edge[out=0,in=180,->,shorten >=.5ex, shorten <=.5ex] (5,-1);
		%\end{uncoverenv}
	\end{tikzpicture} \\
	\only<1>{All local variables of functions are placed at the \textit{stack}. \\
		It grows and shrinks as variables are declared and functions return.}
	\only<2>{Dynamical memory is allocated on the \textit{heap}. \\
		The example shows a function with two local \textit{int} variables.}
	\begin{onlyenv}<3>
		\begin{lstlisting}[numbers=none]
malloc(sizeof(int));
\end{lstlisting}
	Reserves exactly the amount of memory an \textit{int} variable takes.
	\end{onlyenv}
	\begin{onlyenv}<4>
		\begin{lstlisting}[numbers=none]
int *new_block = malloc(sizeof(int));
\end{lstlisting}
	The adress of that memory block is stored in an \textit{int} pointer.
	\end{onlyenv}
	\only<5>{\textit{malloc()} just needs to know the size of the block it reserves. \\
		Let us allocate a \textit{long} variable as well.}
\end{frame}
\begin{frame}[fragile]{\textit{malloc()} in detail}
	The function declaration might be a little bit confusing:
	\begin{lstlisting}[numbers=none]
void *malloc(size_t size);
\end{lstlisting}
	\begin{itemize}
		\item \textit{size\_t} is an \textbf{unsigned integer} type. \\
		Any positive integer number (e.g. an \textit{int} $>0$) will do the job.
		\item \textit{size} is the size of the reserved block in \textbf{bytes}. \\
		If you want to use that block \textit{seriously}, pass the size of an actual type (e.g. \textit{sizeof(int)}).
		\item A \textit{void} pointer is returned since \textit{malloc()} does not know how you want to use the reserved block. By assigning it to a regular pointer variable it is automatically converted to that type.
	\end{itemize}
\end{frame}

\begin{frame}[fragile]{Casting or not casting\dots}
	Some people claim you had to explicitly \textit{cast} the return value of \textit{malloc()}.\\
	\ \\
	\only<2->{\textbf{This is outdated.}} \ \\
	\ \\
	\begin{onlyenv}<3->
		\begin{lstlisting}[numbers=none]
int *block = malloc((sizeof(int));
\end{lstlisting}
	has the same result as
	\begin{lstlisting}[numbers=none]
int *block = (int*) malloc((sizeof(int));
\end{lstlisting}
	while the second one contains a redundant \textit{cast} and if you want to change the type of \textit{block} later, you will have to hit more keys. Consider:
	\begin{lstlisting}[numbers=none]
int *block = malloc(sizeof *block); /* gold standard */
\end{lstlisting}
	\end{onlyenv}
\end{frame}

\begin{frame}{\dots that is the question}
	\huge
	\centering
	Confused? \\
	\ \\
	\only<2->{\large Do not typecast the result of malloc() \& Co.}
\end{frame}
\begin{frame}[fragile]{Tidying up}
	Unlike normally declared variables, dynamically allocated storage is not automatically released when the function returns.
	\begin{lstlisting}
void foo() {
	int *bar = malloc(sizeof(int));
}
\end{lstlisting}
	With the pointer \textit{bar} being removed from the stack, we havo no reference on its allocated memory and those four bytes are blocked forever! \\
	\ \\
	\begin{lstlisting}[numbers=none]
free(void *ptr);
\end{lstlisting}
	Pass any pointer to previously allocated memory to \textit{free()} and it gets realeased. If you pass pointers on other things, undefined behaviour occurs (most likely program crashes).
\end{frame}

\section{Dynamic arrays}
\subsection{}
\begin{frame}[fragile]{Reserving large chunks}
	To get a dynamic array of a certain \textit{type} and \textit{length}, you have to
	\begin{itemize}
		\item Pass the block size $length * sizeof(type)$ to \textit{malloc()}
		\item Assign the return value to a pointer to \textit{type}
	\end{itemize}\ \\
	\ \\
	\textit{int} array with 42 elements:
	\begin{lstlisting}[numbers=none]
int *field = malloc(42 * sizeof(*field));
\end{lstlisting}\ \\
	\ \\
	Since the size of your dynamically allocated array is unknown at compile time, you cannot use \textit{sizeof} to get its length. Save it in its own variable!\\
	\ \\
	With the help of pointer arithmetic, you can use the dynamic array like a "normal" one.
\end{frame}

\begin{frame}[fragile]{The fancy alternative}
		\begin{lstlisting}[numbers=none]
void *calloc(size_t nmemb, size_t size);
\end{lstlisting}
	\begin{itemize}
		\item Allocates a block of $nmemb * size$ bytes, where \textit{nmemb} is supposed to be the array's length and \textit{size} the size its type.
		\item The whole block is filled with \textit{0}s
	\end{itemize}
	\ \\
	\begin{lstlisting}[numbers=none]
int field_length = 42;
int *field = malloc(field_length * sizeof(*field));
for (int i = 0; i < field_length; i++)
	field[i] = 0;
\end{lstlisting}
	\center	$\downarrow$ Feel the difference $\downarrow$
	\begin{lstlisting}[numbers=none]
int field_length = 42;
int *field = calloc(field_length, sizeof(*field));
\end{lstlisting}
\end{frame}

\begin{frame}[fragile]{Resizing arrays}
	Now we come to the point that motivated us to use dynamic arrays:\\
	\ \\
	\begin{lstlisting}[numbers=none]
void *realloc(void *ptr, size_t size);
\end{lstlisting}
	\begin{itemize}
		\item \textit{ptr} is a pointer to a dynamically allocated memory block
		\item \textit{size} is the wanted new size of the memory block
		\item The return value is a pointer to the resized block
	\end{itemize} \ \\
	\ \\
	Note that the new \textit{size} can be greater or smaller than the old one!
	\begin{itemize}
		\item If it's smaller, you may lose some data at the end of the block
		\item If it's greater, the block may be at a different location in the memory $\rightarrow$ \textit{ptr} is freed then, also the additional bytes are not initialized
	\end{itemize}
\end{frame}
\begin{frame}[fragile]{Clean up your code}
	Passing arrays between functions can be complicated if you store the pointer and the length seperataly.\\
	\ \\
	Do you remember a way to keep different things together?\\
	\ \\
	\begin{uncoverenv}<2->
		\begin{lstlisting}[numbers=none]
struct int_array {
	int *field;
	int length;
}
\end{lstlisting}\ \\
		\ \\
		This allows you to use the \textit{struct int\_array} as a single argument or return value. Even better: pass a pointer on that structure.
	\end{uncoverenv}
\end{frame}
\begin{frame}[fragile]{Strings from pointers to \textit{char}}
	By handling strings as dynamic \textit{char} arrays you can alter their size which is needed for many operations on them.
	\begin{itemize}
		\item \textit{strlen()} returns the actual length of a string (up to '\textbackslash 0' character)
		\item \textit{strncpy()} copies a string into a dynamically allocated block
	\end{itemize}\ \\
	\ \\
	\begin{lstlisting}[numbers=none]
char conststr[] = "Hello";			   /* not of much use */
int bufsize = strlen(conststr) + 1;	   /* add '\0' char */
char *str = calloc(bufsize, sizeof(*str));
str = strncpy(str, conststr, bufsize); /* ready to go */
\end{lstlisting}\ \\
	\ \\
	These functions and others are declared in \textit{string.h}.
	\begin{lstlisting}[numbers=none]
$ man string.h
\end{lstlisting}
\end{frame}
\section{Exercise}
\subsection{}
\begin{frame}[fragile]{String concatenation}
	\textit{strncat()} concatenates two strings. Have a closer look at it:
	\begin{lstlisting}[numbers=none]
$ man 3 strncat
\end{lstlisting}
	\begin{itemize}
		\item Write a program that reads a series of strings from the user input and concatenates them.
		\item Each string is put at the front so that the result is in reversed order.
		\begin{itemize}
			\item<2-> Hint: End the input phase when a '\textbackslash n' is read (empty line).
			\item<3-> Hint: Always check if your buffer is large enough and resize it, if needed.
		\end{itemize}
		\item \textbf{Experts}: At the end, let the user enter one last string. Check, if that one occures in the string you have put together.
		\begin{itemize}
			\item<4-> Hint: \textit{strstr()} may be an option.
		\end{itemize}
	\end{itemize}		
\end{frame}
\begin{frame}{Vector operations}
	\begin{itemize}
		\item Write a program that takes two vectors as input and prints their sum.
		\item The number of elements in each vector is up to the user.
		\begin{itemize}
			\item<2-> Hint: You need some conventions to allow the user to tell you when the input of a vector is finished.
			\item<3-> Hint: Think of possible input errors, e.g. one vector having more elements.
		\end{itemize}
		\item \textbf{Experts}: do the same with two matrices
		\begin{itemize}
			\item<4-> Hint: A 2D dynamic array is a dynamic array of pointers. Each of those pointers has its own dynamic array allocated to it.
			\item<5-> Hint: You need a lot more conventions!
		\end{itemize}
	\end{itemize}
\end{frame}
\begin{frame}[fragile]{Hangman}
	Write a simple hangman game! Everytime the player is wrong,
	add a new part to the gallows / the hanging man and print it to the terminal:
	\begin{lstlisting}[numbers=none]
       -----    
       |/  o 
       |  ^W^
       |  / \		word: game o_e_
       |     		wrong: hfnwitc
     -----   
    /     \
   /       \
===============
\end{lstlisting}
	\textbf{Experts}: Have a list of possible words in the source code and let the program randomly choose one at the start.\\
	\ \\
	Hint: \textit{strncmp()} compares two strings.
\end{frame}
% nothing to do from here on
\end{document}
