%% Nothing to modify here.
%% make sure to include this before anything else

\documentclass[10pt]{beamer}
\usetheme{Szeged}

% packages
\usepackage{color}
\usepackage{listings}

% color definitions
\definecolor{mygreen}{rgb}{0,0.6,0}
\definecolor{mygray}{rgb}{0.5,0.5,0.5}
\definecolor{mymauve}{rgb}{0.58,0,0.82}

% re-format the title frame page
\makeatletter
\def\supertitle#1{\gdef\@supertitle{#1}}%
\setbeamertemplate{title page}
{
  \vbox{}
  \vfill
  \begin{centering}
  \begin{beamercolorbox}[sep=8pt,center]{title}
      \usebeamerfont{supertitle}\@supertitle
   \end{beamercolorbox}
    \begin{beamercolorbox}[sep=8pt,center]{title}
      \usebeamerfont{title}\inserttitle\par%
      \ifx\insertsubtitle\@empty%
      \else%
        \vskip0.25em%
        {\usebeamerfont{subtitle}\usebeamercolor[fg]{subtitle}\insertsubtitle\par}%
      \fi%     
    \end{beamercolorbox}%
    \vskip1em\par
    \begin{beamercolorbox}[sep=8pt,center]{author}
      \usebeamerfont{author}\insertauthor
    \end{beamercolorbox}
    \begin{beamercolorbox}[sep=8pt,center]{institute}
      \usebeamerfont{institute}\insertinstitute
    \end{beamercolorbox}
    \begin{beamercolorbox}[sep=8pt,center]{date}
      \usebeamerfont{date}\insertdate
    \end{beamercolorbox}\vskip0.5em
    {\usebeamercolor[fg]{titlegraphic}\inserttitlegraphic\par}
  \end{centering}
  \vfill
}
\makeatother

% insert frame number
\expandafter\def\expandafter\insertshorttitle\expandafter{%
      \insertshorttitle\hfill%
\insertframenumber\,/\,\inserttotalframenumber}

% preset-listing options
\lstset{
  backgroundcolor=\color{white},   
  % choose the background color; 
  % you must add \usepackage{color} or \usepackage{xcolor}
  basicstyle=\footnotesize,        
  % the size of the fonts that are used for the code
  breakatwhitespace=false,         
  % sets if automatic breaks should only happen at whitespace
  breaklines=true,                 % sets automatic line breaking
  captionpos=b,                    % sets the caption-position to bottom
  commentstyle=\color{mygreen},    % comment style
  % deletekeywords={...},            
  % if you want to delete keywords from the given language
  extendedchars=true,              
  % lets you use non-ASCII characters; 
  % for 8-bits encodings only, does not work with UTF-8
  frame=single,                    % adds a frame around the code
  keepspaces=true,                 
  % keeps spaces in text, 
  % useful for keeping indentation of code 
  % (possibly needs columns=flexible)
  keywordstyle=\color{blue},       % keyword style
  % morekeywords={*,...},            
  % if you want to add more keywords to the set
  numbers=left,                    
  % where to put the line-numbers; possible values are (none, left, right)
  numbersep=5pt,                   
  % how far the line-numbers are from the code
  numberstyle=\tiny\color{mygray}, 
  % the style that is used for the line-numbers
  rulecolor=\color{black},         
  % if not set, the frame-color may be changed on line-breaks 
  % within not-black text (e.g. comments (green here))
  stepnumber=1,                    
  % the step between two line-numbers. 
  % If it's 1, each line will be numbered
  stringstyle=\color{mymauve},     % string literal style
  tabsize=4,                       % sets default tabsize to 4 spaces
  title=\lstname                   
  % show the filename of files included with \lstinputlisting; 
  % also try caption instead of title
}

% macro for code inclusion
\newcommand{\includecode}[2][c]{
	\lstinputlisting[caption=#2, style=custom#1]{#2}
}	% nothing to do here
\usepackage[english]{babel}

\usepackage[utf8]{inputenc}

\newcommand{\course}{
	C introduction
}

\author{
	Richard Mörbitz,
	Manuel Thieme
}

\lstset{
	language = C,
	showspaces = false,
	showtabs = false,
	showstringspaces = false,
	tabsize = 4,
	escapechar = @
}

% meta-information
\newcommand{\topic}{
	Pointers
}
\usepackage{tikz}
\usepackage[absolute,overlay]{textpos}
\usetikzlibrary{arrows}
\tikzset{arrow/.style={-latex, shorten >=.5ex, shorten <=.5ex}}
\lstset{
  moredelim=**[is][\only<+(1)>{\color{red}}]{§}{§},
}
% nothing to do here
\title{\topic}
\supertitle{\course}
\date{}

% the actual document
\begin{document}

\maketitle

\begin{frame}{Contents}
	\tableofcontents
\end{frame}

\section{Motivation}
\subsection{}
\begin{frame}[fragile]{RGB}
	Consider a function that calculates the RGB values of a hex color string:
	\begin{lstlisting}[numbers=none]
int calcRGB(char hexString[]) {
	... 		/* converting hexString into RGB values */
	return ???;
}
\end{lstlisting}
	\begin{itemize}
		\item It is not possible to return 3 values.
	\end{itemize}
	We could write 3 different functions:
	\begin{lstlisting}[numbers=none]
int calcR(char hexString[]) { ... }	/* returns R value */
int calcG(char hexString[]) { ... }	/* returns G value */
int calcB(char hexString[]) { ... }	/* returns B value */
\end{lstlisting}
	Or we declare the 3 variables before the function call and just tell the function were to put the values.
\end{frame}
\begin{frame}{Memory again}
	\begin{itemize}[<+->]
		\item You have two int variables in your main function.
		\item Now you call a function
		\item You want to change the value of a variable in the main scope
	\end{itemize}
	\begin{textblock}{10}(1,6.5)
		\begin{tikzpicture}[font=\scriptsize,x=2.5cm]
			
			\draw (0,1) -- (4,1);
			\draw (0,1) -- (0,1.3);
			\draw (0,1.3) -- (4,1.3);
			\draw (4,1) -- (4,1.3);
			
			\node[above=.6em] at (0,1) {\#0};
			\draw[dashed] (.5,1) -- (.5,1.3);
			\node[above=.6em] at (.5,1) {\#4};
			\draw[dashed] (1,1) -- (1,1.3);
			\node[above=.6em] at (1,1) {\#8};
			\draw[dashed] (1.5,1) -- (1.5,1.3);
			\node[above=.6em] at (1.5,1) {\#12};
			\draw[dashed] (2,1) -- (2,1.3);
			\node[above=.6em] at (2,1) {\#16};
			\draw[dashed] (2.5,1) -- (2.5,1.3);
			\node[above=.6em] at (2.5,1) {\#20};
			\draw[dashed] (3,1) -- (3,1.3);
			\node[above=.6em] at (3,1) {\#24};
			\draw[dashed] (3.5,1) -- (3.5,1.3);
			\node[above=.6em] at (3.5,1) {\#28};
			
			\node[blue, below=.4em, right=0em] at (0.15,1.3) {int};
			\draw[dashed, blue] (0,1) -- (.5,1);
			\draw[dashed, blue] (.5,1) -- (.5,1.3);
			\draw[dashed, blue] (0,1.3) -- (.5,1.3);
			\draw[dashed, blue] (0,1) -- (0,1.3);
		
			\node[blue, below=.4em, right=0em] at (.65,1.3) {int};
			\draw[dashed, blue] (.5,1) -- (1,1);
			\draw[dashed, blue] (1,1) -- (1,1.3);
			\draw[dashed, blue] (.5,1.3) -- (1,1.3);
			\draw[dashed, blue] (.5,1) -- (.5,1.3);
			
			
			\draw[orange, dashed] (-.2,.6) -- (-.2,1.9) node[right=1.7em, above=0em]{Main scope};
			\draw[orange, dashed] (-.2,.6) -- (1,.6);
			\draw[orange, dashed] (1,.6) -- (1,1.9);
			\draw[orange, dashed] (-.2,1.9) -- (1,1.9);		
			
			\begin{uncoverenv}<5->
				\node[teal, below=.4em, right=0em] at (1.4,1.3) {int*};
				\draw[dashed, teal] (1,1) -- (2,1);
				\draw[dashed, teal] (2,1) -- (2,1.3);
				\draw[dashed, teal] (1,1.3) -- (2,1.3);
				\draw[dashed, teal] (1,1) -- (1,1.3);
				\draw (1.5,1) edge[teal,out=220,in=0,shorten >=0ex,shorten <=.5ex] (1.2,.7);
			\end{uncoverenv}		
			
			\begin{uncoverenv}<6->
				\node[teal, below=.4em, right=0em] at (2.4,1.3) {int*};
				\draw[dashed, teal] (2,1) -- (3,1);
				\draw[dashed, teal] (3,1) -- (3,1.3);
				\draw[dashed, teal] (2,1.3) -- (3,1.3);
				\draw[dashed, teal] (2,1) -- (2,1.3);
			
				\node[teal, below=.4em, right=0em] at (3.4,1.3) {int*};
				\draw[dashed, teal] (3,1) -- (4,1);
				\draw[dashed, teal] (4,1) -- (4,1.3);
				\draw[dashed, teal] (3,1.3) -- (4,1.3);
				\draw[dashed, teal] (3,1) -- (3,1.3);
			\end{uncoverenv}		
			
			\begin{uncoverenv}<6->
				\draw[arrow, teal, bend angle=45, bend left] (2.5,1) to (.25,1);
				\draw[arrow, teal, bend angle=50, bend left] (3.5,1) to (.25,1);
			\end{uncoverenv}		
			
			\begin{uncoverenv}<6->
				\draw (.15,.9) -- (.15,.6) node[below]{\textit{int a;}};
				\draw (.75,.9) -- (.75,.6) node[below]{\textit{int b;}};
				\draw (1.5,.9) -- (1.5,.6) node[below]{\textit{int$^*$ pb = \&b;}};
				\draw (2.6,.9) -- (2.6,.6) node[below]{\textit{int$^*$ pa = \&a;}};
				\draw (3.8,.9) -- (3.8,.6) node[below]{\textit{int$^*$ ndpa = \&a;}};
			\end{uncoverenv}
			
			\begin{uncoverenv}<2->
				\draw[red, dashed] (1,.6) -- (1,1.9) node[right=2.3em, above=0em]{Function scope};
				\draw[red, dashed] (1,.6) -- (4.2,.6);
				\draw[red, dashed] (4.2,.6) -- (4.2,1.9);
				\draw[red, dashed] (1,1.9) -- (4.2,1.9);
			\end{uncoverenv}
			
			\draw<3-> (1.2,.7) edge[teal,out=180,in=320,arrow,shorten >=.5ex,shorten <=0ex] (.75,1);
		\end{tikzpicture}
	\end{textblock}
	\ \\\ \\\ \\\ \\\ \\\ \\\ \\\ \\\ \\
	\begin{itemize}[<+->]
		\item You'll have to pass the address of this variable
		\item This address is stored in a \textit{pointer} variable
		\item This method is called \textit{call by reference}
	\end{itemize}
\end{frame}
\section{Pointers}
\subsection{}
\begin{frame}[fragile]{Operators}
	\begin{itemize}
		\item To declare a Pointer, use the \textit{dereference operator} \textbf{*}
		\item To get the address of a variable, C comes with the \textit{adress operator} \textbf{\&}
		\item To access the variable a pointer points to, dereference it with the \textit{dereference operator} \textbf{*}
	\end{itemize}
	\begin{lstlisting}[numbers=none]
int a = 42;
int* pa;	/* declare an int pointer*/
pa = &a;	/* initialize pa as pointer to a */
*pa = 13;	/* change a */
\end{lstlisting}
\end{frame}
\begin{frame}[fragile]{incrementing and decrementing}
	If you want to increment or decrement the variable a pointer points to, you have to use Parentheses.
	\begin{lstlisting}[numbers=none]
int a = 42;
int* pa = &a;	/* define pa as pointer to a */
(*pa)++;		/* increment a */
(*pa)--;		/* decrement a */
\end{lstlisting}
\ \\\ \\
If you had not used the parentheses, you would have in-/decremented the pointer, not the variable it points to. Congratulations, you just invented pointer arithmetic but we will talk later about that.
\end{frame}
\begin{frame}[fragile]{Back to RGB}
	Now we can think of the RGB function as one function, taking the hexString and 3 Pointers:
	\begin{lstlisting}[numbers=none]
void calcRGB(char hexString[], int* r, int* g, int* b) {
	...
	*r = calculatedRValue;
	*g = calculatedGValue;
	*b = calculatedBValue;
}
\end{lstlisting}
	Call it with
	\begin{lstlisting}[numbers=none]
int r, g, b;
calcRGB("ffffff", &r, &g, &b);
\end{lstlisting}
	\begin{itemize}
		\item You now should understand how scanf works.
	\end{itemize}
\end{frame}
\begin{frame}[fragile]{Returning pointers}
Pointers can be return values, too.\\\ \\
\textbf{But} 
	\begin{lstlisting}[numbers=none]
int* someFunction() {	
	int a = 42;
	return &a;
}
\end{lstlisting}
	\begin{itemize}
		\item Dafuq did just happen?
	\end{itemize}
\end{frame}
\section{Exercise}
\subsection{}
\begin{frame}{Exercises}
	\begin{itemize}
		\item You are now able to solve tasks 22 and 23.
	\end{itemize}
\end{frame}
\section{Pointer arithmetic}
\subsection{}
\begin{frame}[fragile]{p++}
	You can in-/decrement a pointer. If you do so, the address it points to will change.\\
	The address changes by the size of the pointer type.
	\begin{columns}[T]
		\column{.5\textwidth}
		\begin{lstlisting}[numbers=none]
int a, b;
long c;
int* p = §&a§;
§p++§;
§p++§;
§p++§;
\end{lstlisting}
		\column{.5\textwidth}
		\begin{tikzpicture}[font=\scriptsize,x=2.5cm]
				
			\draw (0,1) -- (2,1);
			\draw (0,1) -- (0,1.3);
			\draw (0,1.3) -- (2,1.3);
			\draw (2,1) -- (2,1.3);
			
			\node[above=.6em] at (0,1) {\#0};
			\draw[dashed] (.5,1) -- (.5,1.3);
			\node[above=.6em] at (.5,1) {\#4};
			\draw[dashed] (1,1) -- (1,1.3);
			\node[above=.6em] at (1,1) {\#8};
			\draw[dashed] (1.5,1) -- (1.5,1.3);
			\node[above=.6em] at (1.5,1) {\#12};
			\draw[dashed] (2,1) -- (2,1.3);
			
			\node[blue, below=-.15em] at (0.25,1.3) {int a};
			\node[blue, below=-.15em] at (.75,1.3) {int b};
			\node[teal, below=-.15em] at (1.5,1.3) {long c};
			\begin{uncoverenv}<2>
				\draw (2.1,1.2) edge[teal,out=225,in=315,arrow,shorten >=.5ex,shorten <=.5ex] (0,1);
			\end{uncoverenv}
			\begin{uncoverenv}<3>
				\draw (2.1,1.2) edge[teal,out=225,in=315,arrow,shorten >=.5ex,shorten <=.5ex] (.5,1);
			\end{uncoverenv}
			\begin{uncoverenv}<4>
				\draw (2.1,1.2) edge[teal,out=225,in=315,arrow,shorten >=.5ex,shorten <=.5ex] (1,1);
			\end{uncoverenv}
			\begin{uncoverenv}<5>
				\draw (2.1,1.2) edge[teal,out=225,in=315,arrow,shorten >=.5ex,shorten <=.5ex] (1.5,1);
			\end{uncoverenv}
		\end{tikzpicture}
	\end{columns}
	\begin{itemize}
		\item<5-> Since the pointer is of type \textit{int*}, the target adress moves only the size of \textit{int}
	\end{itemize}
\end{frame}
\begin{frame}[fragile]{Pointer and Arrays}
	The identifier of an array can be considered a pointer.\\
	This means we can consider the index as an offset for the pointer and access array elements trough pointer arithmetic:
	\begin{lstlisting}[numbers=none]
int leet[4] = {1, 3, 3, 7};
int* pleet = leet;
*(pleet++) = 2;
printf("%d %d\n", *pleet, *(pleet + 2));
\end{lstlisting}
	\begin{itemize}
		\item What is the output?
	\end{itemize}
	\begin{uncoverenv}<2->
		\begin{lstlisting}[numbers=none]
2 7
\end{lstlisting}
		\begin{itemize}
			\item Why?
			\begin{itemize}
				\item<3-> Hint: Wasn't there a difference between \textit{c++} and \textit{++c} ?
			\end{itemize}
		\end{itemize}
	\end{uncoverenv}
\end{frame}
\begin{frame}[fragile]{\textbf{argc} and \textbf{argv}}
	You can pass strings to the main function by writing them on the command line.
	\begin{lstlisting}[numbers=none]
$ ./a.out string1 longer_string2
\end{lstlisting}
	\begin{itemize}
		\item They are stored in \textit{argv}\footnote{Short for \textit{argument value}}
		\item \textit{argv} is an array of pointers to the first character of a string
		\item \textbf{Caution:} \textit{argv[0]} is the name by which you called the program
		\item \textit{argc}\footnote{Short for \textit{argument count}} is the number of strings stored in \textit{argv}
	\end{itemize}
\end{frame}
\section{Exercise}
\subsection{}
\begin{frame}{Exercises}
	\begin{itemize}
		\item You are now able to solve tasks 24 and 25.
	\end{itemize}
\end{frame}
% TODO from here on: build your own content


% nothing to do from here on
\end{document}
