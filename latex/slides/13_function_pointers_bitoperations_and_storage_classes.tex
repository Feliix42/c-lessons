%% Nothing to modify here.
%% make sure to include this before anything else

\documentclass[10pt]{beamer}
\usetheme{Szeged}

% packages
\usepackage{color}
\usepackage{listings}

% color definitions
\definecolor{mygreen}{rgb}{0,0.6,0}
\definecolor{mygray}{rgb}{0.5,0.5,0.5}
\definecolor{mymauve}{rgb}{0.58,0,0.82}

% re-format the title frame page
\makeatletter
\def\supertitle#1{\gdef\@supertitle{#1}}%
\setbeamertemplate{title page}
{
  \vbox{}
  \vfill
  \begin{centering}
  \begin{beamercolorbox}[sep=8pt,center]{title}
      \usebeamerfont{supertitle}\@supertitle
   \end{beamercolorbox}
    \begin{beamercolorbox}[sep=8pt,center]{title}
      \usebeamerfont{title}\inserttitle\par%
      \ifx\insertsubtitle\@empty%
      \else%
        \vskip0.25em%
        {\usebeamerfont{subtitle}\usebeamercolor[fg]{subtitle}\insertsubtitle\par}%
      \fi%     
    \end{beamercolorbox}%
    \vskip1em\par
    \begin{beamercolorbox}[sep=8pt,center]{author}
      \usebeamerfont{author}\insertauthor
    \end{beamercolorbox}
    \begin{beamercolorbox}[sep=8pt,center]{institute}
      \usebeamerfont{institute}\insertinstitute
    \end{beamercolorbox}
    \begin{beamercolorbox}[sep=8pt,center]{date}
      \usebeamerfont{date}\insertdate
    \end{beamercolorbox}\vskip0.5em
    {\usebeamercolor[fg]{titlegraphic}\inserttitlegraphic\par}
  \end{centering}
  \vfill
}
\makeatother

% insert frame number
\expandafter\def\expandafter\insertshorttitle\expandafter{%
      \insertshorttitle\hfill%
\insertframenumber\,/\,\inserttotalframenumber}

% preset-listing options
\lstset{
  backgroundcolor=\color{white},   
  % choose the background color; 
  % you must add \usepackage{color} or \usepackage{xcolor}
  basicstyle=\footnotesize,        
  % the size of the fonts that are used for the code
  breakatwhitespace=false,         
  % sets if automatic breaks should only happen at whitespace
  breaklines=true,                 % sets automatic line breaking
  captionpos=b,                    % sets the caption-position to bottom
  commentstyle=\color{mygreen},    % comment style
  % deletekeywords={...},            
  % if you want to delete keywords from the given language
  extendedchars=true,              
  % lets you use non-ASCII characters; 
  % for 8-bits encodings only, does not work with UTF-8
  frame=single,                    % adds a frame around the code
  keepspaces=true,                 
  % keeps spaces in text, 
  % useful for keeping indentation of code 
  % (possibly needs columns=flexible)
  keywordstyle=\color{blue},       % keyword style
  % morekeywords={*,...},            
  % if you want to add more keywords to the set
  numbers=left,                    
  % where to put the line-numbers; possible values are (none, left, right)
  numbersep=5pt,                   
  % how far the line-numbers are from the code
  numberstyle=\tiny\color{mygray}, 
  % the style that is used for the line-numbers
  rulecolor=\color{black},         
  % if not set, the frame-color may be changed on line-breaks 
  % within not-black text (e.g. comments (green here))
  stepnumber=1,                    
  % the step between two line-numbers. 
  % If it's 1, each line will be numbered
  stringstyle=\color{mymauve},     % string literal style
  tabsize=4,                       % sets default tabsize to 4 spaces
  title=\lstname                   
  % show the filename of files included with \lstinputlisting; 
  % also try caption instead of title
}

% macro for code inclusion
\newcommand{\includecode}[2][c]{
	\lstinputlisting[caption=#2, style=custom#1]{#2}
}	% nothing to do here
%% Fill in metadata here that do not change over the course
%% They all are marked with the term "TODO". 
%% Search functions usually do the trick

% TODO select the targeted language
\usepackage[english]{babel}
% \usepackage[ngerman]{babel}

% TODO select the encoding
\usepackage[utf8]{inputenc}
% usepackage[latin1]{inputenc}

\newcommand{\course}{
	% TODO replace this with the name of the course
	cadvanced
}

\author{
	% TODO fill in the authors name
	Tony Template
}

\lstset{
	% TODO adapt these settings to your mainly used language
	% also see http://en.wikibooks.org/wiki/LaTeX/Source_Code_Listings
	% NOTE you can override these settings in individual cases 
	language = C,
	showspaces = false,
	showtabs = false,
	showstringspaces = false,
	escapechar = @
} % TODO modify this if you have not already done so

% meta-information
\newcommand{\topic}{
	Bitoperations, storage classes, macros and function pointers
}

% nothing to do here
\title{\topic}
\supertitle{\course}
\date{}

% the actual document
\begin{document}

\maketitle

\begin{frame}{Contents}
	\tableofcontents
\end{frame}

\section{Bitoperations}
\subsection{}

\begin{frame}{About 0 and 1}
	As you know, your computer stores data in 0s and 1s.\\
	With C beeing a very low level language, we can manipulate data on bit layer.\\\ \\
	
	C offers the following bitoperations:\\\ \\
	
	\begin{tabular}{|l|c|l|}
																						  	  \hline
		\textbf{Operation} 	& \textbf{Syntax} 	& \textbf{Example} 							\\\hline
		Or 					& $|$ 				& $4\ |\ 1 = 0100\ |\ 0001 = 0101 = 5$ 		\\\hline
		And 				& $\&$ 				& $6\ \&\ 2 = 0110\ \&\ 0010 = 0010 = 2$ 	\\\hline
		Shift-Left 			& $<<$ 				& $5 << 2 = 00101 << 2 = 10100 = 20$ 		\\\hline
		Shift-Right 		& $>>$ 				& $10 >> 2 = 1010 >> 2 = 0010 = 2$ 			\\\hline
	\end{tabular}
\end{frame}

\section{Storage classes}
\subsection{}

\begin{frame}{ToDo}

\end{frame}

\section{Macros}
\subsection{}

\begin{frame}{ToDo}

\end{frame}

\section{Pointers to functions}
\subsection{}

\begin{frame}{Passing functions to functions}
	It is possible to pass a function as an argument of another function.\\
	Doing so, you have a function that can call different other functions.\\
	To be a parameter, there have to be information about the return type and the parameter list.\\
	\ \\
	Actually you do not pass a function but a pointer to that function.
\end{frame}

\begin{frame}[fragile]{Syntax}
	\begin{lstlisting}
#include <stdio.h>

int add(int a, int b) {
	return a + b;
}

int sub(int a, int b) {
	return a - b;
}

void printFunc(int (*f)(int, int), int a, int b) {
	 printf("%d\n",f(a, b));
}

int main(void) {
	printFunc(add, 1, 2);
	printFunc(sub, 1, 2);
	return 0;
}
\end{lstlisting}

\end{frame}

\begin{frame}{Mapping}
	Pointers to functions are often used for mapping. If you want to iterate through a list and call a function for every list item, you could do this with a new loop each time, but you also could write a function that takes a list and a function.\\
	\ \\
	If only we had an example to try this out...\\
	Oh wait... we do: The Dungeon.\\\ \\
	\begin{itemize}
		\item Write a mapping function that takes a list of monsters and a function.
		\item Edit the print\_monster\_list function to a print\_entity function.
		\item Now you can print the monsters list by calling the mapping function passing the list and the print\_entity function.
	\end{itemize}
\end{frame}

% nothing to do from here on
\end{document}
