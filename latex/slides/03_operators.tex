%% Nothing to modify here.
%% make sure to include this before anything else

\documentclass[10pt]{beamer}
\usetheme{Szeged}

% packages
\usepackage{color}
\usepackage{listings}

% color definitions
\definecolor{mygreen}{rgb}{0,0.6,0}
\definecolor{mygray}{rgb}{0.5,0.5,0.5}
\definecolor{mymauve}{rgb}{0.58,0,0.82}

% re-format the title frame page
\makeatletter
\def\supertitle#1{\gdef\@supertitle{#1}}%
\setbeamertemplate{title page}
{
  \vbox{}
  \vfill
  \begin{centering}
  \begin{beamercolorbox}[sep=8pt,center]{title}
      \usebeamerfont{supertitle}\@supertitle
   \end{beamercolorbox}
    \begin{beamercolorbox}[sep=8pt,center]{title}
      \usebeamerfont{title}\inserttitle\par%
      \ifx\insertsubtitle\@empty%
      \else%
        \vskip0.25em%
        {\usebeamerfont{subtitle}\usebeamercolor[fg]{subtitle}\insertsubtitle\par}%
      \fi%     
    \end{beamercolorbox}%
    \vskip1em\par
    \begin{beamercolorbox}[sep=8pt,center]{author}
      \usebeamerfont{author}\insertauthor
    \end{beamercolorbox}
    \begin{beamercolorbox}[sep=8pt,center]{institute}
      \usebeamerfont{institute}\insertinstitute
    \end{beamercolorbox}
    \begin{beamercolorbox}[sep=8pt,center]{date}
      \usebeamerfont{date}\insertdate
    \end{beamercolorbox}\vskip0.5em
    {\usebeamercolor[fg]{titlegraphic}\inserttitlegraphic\par}
  \end{centering}
  \vfill
}
\makeatother

% insert frame number
\expandafter\def\expandafter\insertshorttitle\expandafter{%
      \insertshorttitle\hfill%
\insertframenumber\,/\,\inserttotalframenumber}

% preset-listing options
\lstset{
  backgroundcolor=\color{white},   
  % choose the background color; 
  % you must add \usepackage{color} or \usepackage{xcolor}
  basicstyle=\footnotesize,        
  % the size of the fonts that are used for the code
  breakatwhitespace=false,         
  % sets if automatic breaks should only happen at whitespace
  breaklines=true,                 % sets automatic line breaking
  captionpos=b,                    % sets the caption-position to bottom
  commentstyle=\color{mygreen},    % comment style
  % deletekeywords={...},            
  % if you want to delete keywords from the given language
  extendedchars=true,              
  % lets you use non-ASCII characters; 
  % for 8-bits encodings only, does not work with UTF-8
  frame=single,                    % adds a frame around the code
  keepspaces=true,                 
  % keeps spaces in text, 
  % useful for keeping indentation of code 
  % (possibly needs columns=flexible)
  keywordstyle=\color{blue},       % keyword style
  % morekeywords={*,...},            
  % if you want to add more keywords to the set
  numbers=left,                    
  % where to put the line-numbers; possible values are (none, left, right)
  numbersep=5pt,                   
  % how far the line-numbers are from the code
  numberstyle=\tiny\color{mygray}, 
  % the style that is used for the line-numbers
  rulecolor=\color{black},         
  % if not set, the frame-color may be changed on line-breaks 
  % within not-black text (e.g. comments (green here))
  stepnumber=1,                    
  % the step between two line-numbers. 
  % If it's 1, each line will be numbered
  stringstyle=\color{mymauve},     % string literal style
  tabsize=4,                       % sets default tabsize to 4 spaces
  title=\lstname                   
  % show the filename of files included with \lstinputlisting; 
  % also try caption instead of title
}

% macro for code inclusion
\newcommand{\includecode}[2][c]{
	\lstinputlisting[caption=#2, style=custom#1]{#2}
}	% nothing to do here
\usepackage[english]{babel}

\usepackage[utf8]{inputenc}

\newcommand{\course}{
	C introduction
}

\author{
	Richard Mörbitz,
	Manuel Thieme
}

\lstset{
	language = C,
	showspaces = false,
	showtabs = false,
	showstringspaces = false,
	tabsize = 4,
	escapechar = @
} % TODO modify this if you have not already done so

% meta-information
\newcommand{\topic}{
	Operators
}

% nothing to do here
\title{\topic}
\supertitle{\course}
\date{}

% the actual document
\begin{document}

\maketitle

\begin{frame}{Contents}
	\tableofcontents
\end{frame}

\section{Overview}
\subsection{}
\begin{frame}[fragile]{Let's calc}
	Each time you type in a value or a variable, it gets evaluated as an expression. You can use operators to make new expressions out of existing ones.
	\begin{lstlisting}[numbers=none]
3;				/* value 3 */
int a = 5; 		/* value 5 */
int b = a + 3;	/* value 8 */
\end{lstlisting}
Got the idea?\\
\ \\Operators are for calculations, comparisons and more.
\end{frame}
\section{Arithmetic operators}
\subsection{}
\begin{frame}[fragile]{Back to School}
	\begin{itemize}
		\item +, -, *, / as all of you should know
		\item \% is the modulo (remainder) operator
		\item *, /, \% get evaluated before +, -
		\item Operations in (  ) are of higher precedence
	\end{itemize}
	\ \\\ \\Something to try out and think about:
	\begin{lstlisting}[numbers=none]
char a = 120;	/* 120 */
char b = 2 * a;			/* not 240 */
\end{lstlisting}
\end{frame}
\begin{frame}[fragile]{You see, it's not that easy}
	\begin{itemize}
		\item Variables may overflow
		\item You shall not devide integers by zero
		\item Integer division differs from floating point division
		\item You can use operators between different data types
		\begin{itemize}
			\item[--] mixing different sizes
			\item[--] mixing integer and floating point variables
		\end{itemize}
	\end{itemize}
	\ \\\ \\
	\begin{lstlisting}[numbers=none]
int i1 = 42, i2 = 23;
short s = 13;
float f = 3.14;

i1 / i2;	/* results in 1, not a real division */
i1 + s;		/* int and short, result is int */
i1 / f;		/* result is float, actual division */
\end{lstlisting}
\end{frame}
\begin{frame}[fragile]{Syntatic sugar}
	As you have seen, you can use any expression on the right side of the assignment operator.\\
	This expression often contains the variable it is assigned to.\\\ \\
	To avoid redundancy, C offers the following short forms:\\
	\begin{lstlisting}[numbers=none]
a += 4;		/* a = a + 4; */
a -= 4;		/* a = a - 4; */
a *= b;		/* a = a * b; */
b /= 42;	/* b = b / 42; */
b %= 2;		/* b = b % 2; */
c++;		/* c = c + 1; */
++c;		/* c = c + 1; */
c--;		/* c = c - 1; */
--c;		/* c = c - 1; */
\end{lstlisting}
\end{frame}
\section{Relational operators}
\subsection{}
\begin{frame}{The truth about expressions}
	Expressions can also be evaluated to truth values.\\
	If a value or a variable equals 0, its corresponding truth value is \textit{false}. Otherwise it's \textit{true}.\\
	The representations of \textit{true} and \textit{false} are 0 and 1.\\
	An expression containing relational operators gets evaluated to such a truth value.\\
	\ \\Relational operators:
	\begin{itemize}
		\item $<$, $>$, $<$=, $>$=
		\item == for "equal to"
		\item != for "not equal to"
	\end{itemize}
\end{frame}
\begin{frame}[fragile]{Do not get confused}
	Imagine the following
	\begin{lstlisting}[numbers=none]
(5 < 7) =@\,@= 1;	/* evaluated to 1 */
\end{lstlisting}
\ \\\ \\Why?
\begin{itemize}
	\item<2-> (5 $<$ 7) is \textbf{true} $\rightarrow$ \textbf{1}
	\item<3-> 1 == 1 is \textbf{true} $\rightarrow$ \textbf{1}
\end{itemize}
\end{frame}
\begin{frame}[fragile]{A sign meant...}
	Assignments are expressions that get evaluated and have a truth value, too.
	Consider:
		\begin{lstlisting}[numbers=none]
c = 0;				/* false */
c = 2 * 5;			/* 2 * 5 = 10, c = 10, true */
c = (0 < 1);		/* 0 < 1 = true, c = 1, true */

a = (b =@\,@= (c = d));	/* Wat? */
\end{lstlisting}\ \\
	\ \\\textit{c++} expressions are evaluated before the increment while \textit{++c} increments first (the same applies on C$--$):
	\begin{lstlisting}[numbers=none]
int c = 42;
int a = c++;	/* a is 42 */
int b = ++c;	/* b is 44 */
\end{lstlisting}
\end{frame}
\section{Logical operators}
\subsection{}
\begin{frame}{Boolean arithmetic}
	Truth values can be connected by boolean operators resulting in a new truth value.
	\begin{itemize}
		\item \&\& for AND (results in \textbf{1} if both operands are true, else \textbf{0})
		\item $||$ for OR (results in \textbf{1} if at least one operator is true, else \textbf{0})
		\item ! for NOT (results in \textbf{1} if the operand is false, else \textbf{0})
	\end{itemize}
	\ \\\ \\Precedence order:\\
	\centering
	! \ $>$ \ \&\& \ $>$ \ $||$ 
\end{frame}
\begin{frame}[fragile]{Seems logic}
	\begin{itemize}
		\item How do you get NAND, NOR and XOR?
	\end{itemize}
	\begin{uncoverenv}<2->
		\begin{lstlisting}[numbers=none]
int a, b;
...
!(a && b);	/* NAND */
!(a || b);	/* NOR */
a != b;		/* XOR */
\end{lstlisting}
	\end{uncoverenv}
\end{frame}
\section{Exercise}
\subsection{}
\begin{frame}{Calculator Light}
	\begin{itemize}
		\item Write a program that asks the user for two decimal numbers and prints their sum, difference, product, integer quotient and remainder.
		\item \textbf{Experts}: Implement a "real" division instead.
	\end{itemize}
\end{frame}
\begin{frame}{Geometry}
	\begin{itemize}
		\item Write a program that asks for a radius and calculates the area of the corresponding circle. (Let's assume $\pi$ == 3.14)
		\item \textbf{Experts}: ask for an additional number. Print 1, if the area is below that number, else print zero.
	\end{itemize}
\end{frame}
% nothing to do from here on
\end{document}
