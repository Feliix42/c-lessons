%% Nothing to modify here.
%% make sure to include this before anything else

\documentclass[10pt]{beamer}
\usetheme{Szeged}

% packages
\usepackage{color}
\usepackage{listings}

% color definitions
\definecolor{mygreen}{rgb}{0,0.6,0}
\definecolor{mygray}{rgb}{0.5,0.5,0.5}
\definecolor{mymauve}{rgb}{0.58,0,0.82}

% re-format the title frame page
\makeatletter
\def\supertitle#1{\gdef\@supertitle{#1}}%
\setbeamertemplate{title page}
{
  \vbox{}
  \vfill
  \begin{centering}
  \begin{beamercolorbox}[sep=8pt,center]{title}
      \usebeamerfont{supertitle}\@supertitle
   \end{beamercolorbox}
    \begin{beamercolorbox}[sep=8pt,center]{title}
      \usebeamerfont{title}\inserttitle\par%
      \ifx\insertsubtitle\@empty%
      \else%
        \vskip0.25em%
        {\usebeamerfont{subtitle}\usebeamercolor[fg]{subtitle}\insertsubtitle\par}%
      \fi%     
    \end{beamercolorbox}%
    \vskip1em\par
    \begin{beamercolorbox}[sep=8pt,center]{author}
      \usebeamerfont{author}\insertauthor
    \end{beamercolorbox}
    \begin{beamercolorbox}[sep=8pt,center]{institute}
      \usebeamerfont{institute}\insertinstitute
    \end{beamercolorbox}
    \begin{beamercolorbox}[sep=8pt,center]{date}
      \usebeamerfont{date}\insertdate
    \end{beamercolorbox}\vskip0.5em
    {\usebeamercolor[fg]{titlegraphic}\inserttitlegraphic\par}
  \end{centering}
  \vfill
}
\makeatother

% insert frame number
\expandafter\def\expandafter\insertshorttitle\expandafter{%
      \insertshorttitle\hfill%
\insertframenumber\,/\,\inserttotalframenumber}

% preset-listing options
\lstset{
  backgroundcolor=\color{white},   
  % choose the background color; 
  % you must add \usepackage{color} or \usepackage{xcolor}
  basicstyle=\footnotesize,        
  % the size of the fonts that are used for the code
  breakatwhitespace=false,         
  % sets if automatic breaks should only happen at whitespace
  breaklines=true,                 % sets automatic line breaking
  captionpos=b,                    % sets the caption-position to bottom
  commentstyle=\color{mygreen},    % comment style
  % deletekeywords={...},            
  % if you want to delete keywords from the given language
  extendedchars=true,              
  % lets you use non-ASCII characters; 
  % for 8-bits encodings only, does not work with UTF-8
  frame=single,                    % adds a frame around the code
  keepspaces=true,                 
  % keeps spaces in text, 
  % useful for keeping indentation of code 
  % (possibly needs columns=flexible)
  keywordstyle=\color{blue},       % keyword style
  % morekeywords={*,...},            
  % if you want to add more keywords to the set
  numbers=left,                    
  % where to put the line-numbers; possible values are (none, left, right)
  numbersep=5pt,                   
  % how far the line-numbers are from the code
  numberstyle=\tiny\color{mygray}, 
  % the style that is used for the line-numbers
  rulecolor=\color{black},         
  % if not set, the frame-color may be changed on line-breaks 
  % within not-black text (e.g. comments (green here))
  stepnumber=1,                    
  % the step between two line-numbers. 
  % If it's 1, each line will be numbered
  stringstyle=\color{mymauve},     % string literal style
  tabsize=4,                       % sets default tabsize to 4 spaces
  title=\lstname                   
  % show the filename of files included with \lstinputlisting; 
  % also try caption instead of title
}

% macro for code inclusion
\newcommand{\includecode}[2][c]{
	\lstinputlisting[caption=#2, style=custom#1]{#2}
}	% nothing to do here
\usepackage[english]{babel}

\usepackage[utf8]{inputenc}

\newcommand{\course}{
	C introduction
}

\author{
	Richard Mörbitz,
	Manuel Thieme
}

\lstset{
	language = C,
	showspaces = false,
	showtabs = false,
	showstringspaces = false,
	tabsize = 4,
	escapechar = @
} % TODO modify this if you have not already done so

% meta-information
\newcommand{\topic}{
	Complex data types
}

% nothing to do here
\title{\topic}
\supertitle{\course}
\date{}

% the actual document
\begin{document}

\maketitle

\begin{frame}{Contents}
	\tableofcontents
\end{frame}

\section{Introduction}
\subsection{}
\begin{frame}[fragile]{Motivation}
	Think of a data type that can store all data belonging to a person:
	\begin{lstlisting}[numbers=none]
char name[];	/* size does not matter (in this example) */
int age, id;
\end{lstlisting}
	However, there seems to be no way to put those different types together. \\ \ \\
	Think of a datatype that can store the state (current color) of traffic lights:
	\begin{lstlisting}[numbers=none]
int color;
/* 0 = red, 1 = yellow, 2 = green */
\end{lstlisting}
	How to avoid someone assigning a different value to \textit{color}?
\end{frame}
\begin{frame}{Limits of primitive data types}
	Primitive data types are fine as long as you want to
	\begin{itemize}
		\item Store a single value that does not depend on other variables
		\item Store a sequence of values of the same type with a constant length \\
		$\rightarrow$ \textit{arrays}
	\end{itemize} \ \\ \ \\
	However, it is not possible to
	\begin{itemize}
		\item Compose variables of different data types to a compound structure \\
		$\rightarrow$ \textit{composite data types}
		\item Have a variable that can only attain certain values \\
		$\rightarrow$ \textit{enumerations}
		\item Have a sequence with an adjustable length \\
		$\rightarrow$ soon...
	\end{itemize}
\end{frame}

\section{Composite data types}
\subsection{}
\begin{frame}[fragile]{Data records}
	Composite data types are derived from primitive data types. You can store any number of primitive variables in one composite variable.
	\begin{itemize}
		\item The composite variable is called \textit{structure} and has the type \textit{struct}
		\item The primitive variables are called \textit{members} of that structure
	\end{itemize} \ \\ \ \\
	Declaration:
	\begin{lstlisting}[numbers=none]
struct person {		/* struct <identifier> */
	int id;
	int age;		/* block for member declaration */
	char name[];
};					/* end declaration with ';' */
\end{lstlisting}
	This defines the new type "struct person".
\end{frame}
\begin{frame}[fragile]{\textit{struct} variables}
	Our new type \textit{struct person} can be used to declare variables any where in its scope:
	\begin{lstlisting}[numbers=none]
struct person pers_alice, pers_bob;
\end{lstlisting} \ \\ \ \\
	You can declare a \textit{struct} variable directly in the type definition:
	\begin{lstlisting}[numbers=none]
struct person {
	/* member declaration */
} pers_alice, pers_bob;
\end{lstlisting}
	If we do not need the struct type \textit{person} for further variable declarations, its identifier can be left out.
\end{frame}
\begin{frame}[fragile]{Definition and member access}
	To initialize the \textit{struct} members upon declaration, enclose the values in braces as we did it for arrays:
	\begin{lstlisting}[numbers=none]
struct person pers_alice = { 1, 20, "Alice" };
\end{lstlisting} \ \\ \ \\
	To access the struct members, use the struct identifier followed by a '\textbf{.}' and the member identifier:
	\begin{lstlisting}[numbers=none]
printf("%d\n", pers_alice.id);
pers_alice.age++;
\end{lstlisting}
\end{frame}
\begin{frame}[fragile]{\textit{struct}s as \textit{struct} members}
	An adress is rather complicated:
	\begin{lstlisting}[numbers=none]
struct adress {
	int postcode;
	/* ... imagine much more members */
};
\end{lstlisting}
	Now, let the \textit{person} have one:
	\begin{lstlisting}[numbers=none]
struct person {
	struct adress contact;
	/* ... and all the other members */
} pers_alice;
\end{lstlisting}
	Access:
\begin{lstlisting}[numbers=none]
pers_alice.contact.postcode = 15430;
\end{lstlisting}
\end{frame}
\begin{frame}[fragile]{\textit{union}s}
	\begin{itemize}
		\item Similar to \textit{struct}s, handle them in the same way
		\item However: only one member can be "active"
		\item If you assign a value to a member, all other members become invalid
	\end{itemize} \ \\ \ \\
	Example:
	\begin{lstlisting}[numbers=none]
union compound {
	int list[3];
	struct {
		int x1, x2, x3;
	} vector;
};
\end{lstlisting}
	$\rightarrow$ could be used as an interface between an \textit{array}-based (\textit{list}) and a \textit{struct}-based (\textit{vector}) implementation.
\end{frame}
\section{Enumerations}
\subsection{}
\begin{frame}[fragile]{Smart aliases}
	An enumeration consists of identifiers that behave like \textit{constant values}. \\
	It is declared using the keyword \textit{enum}:
	\begin{lstlisting}[numbers=none]
enum light {
	RED,
	YELLOW,
	GREEN
};
\end{lstlisting}
	Now you can assign the values \textit{red}, \textit{yellow} and \textit{green} to variables of the type \textit{enum light}. Internally they are represented as numbers ($red = 0$, $yellow = 1$ etc.), but
	\begin{itemize}
		\item Using the aliases is clear and fancy
		\item No invalid values (like \textit{-1}) can be assigned
	\end{itemize}
\end{frame}
\begin{frame}[fragile]{Profit}
	You can determine the values of the constants on your own:
	\begin{lstlisting}[numbers=none]
enum workday {
	MONDAY,			/* 0 */
	TUESDAY,		/* 1 */
	THURSDAY = 3,	/* 3 */
	FRIDAY			/* 4 - implicit (predecessor + 1) */
};
\end{lstlisting}
	However, this can confuse people $\rightarrow$ only use it if there is a good reason. \\ \ \\
	Enumerations provide a nice way to define "global" constants:
	\begin{lstlisting}[numbers=none]
enum { WIDTH = 10, HEIGHT = 20 };
...
char tetris_board[WIDTH][HEIGHT];
\end{lstlisting}
\end{frame}

\section{Style}
\subsection{}
\begin{frame}{Consistency}
	\begin{itemize}
		\item Since complex type definitions heavily rely on blocks, you should use the same coding conventions on them
		\item Let your custom type identifiers start with small letters
	\end{itemize} \ \\ \ \\
	If you define a complex data type, you are very likely going to use it in many different parts of your program. \\
	$\rightarrow$ Have a global type definition, declare the variables in the local context \\ \ \\
	Name \textit{enum} constants in CAPITAL letters to visually seperate them from variables.
\end{frame}
\begin{frame}[fragile]{\textbf{typedef}}
	Sometimes you see people writing code like that:
	\begin{lstlisting}[numbers=none]
typedef struct foo {
	/* member declarations */
} bar;
\end{lstlisting}
	This creates the new type \textit{bar} which is nothing more than a \textit{struct foo}. \\ \ \\
	However, this simple fact is hidden for other programmers working on the same project $\rightarrow$ \textbf{possible confusion}.
	\begin{itemize}
		\item Unclear, if \textit{bar} is a composite type at all
		\item If so, is it a \textit{struct} / \textit{union} / \textit{enum} or something really crazy?
	\end{itemize}
\end{frame}
\begin{frame}{Never use \textit{typedef}.}
	\LARGE
	\centering
	Please, avoid using \textit{typedef}.\footnotemark \ \\ \ \\
	\normalsize
	\flushleft
	\begin{uncoverenv}<2->
		Of course, there are situations in which the use of \textit{typedef} makes sense. BUT:
		\begin{itemize}
			\item Not in the C introduction course
			\item Not for simple \textit{struct}s
		\end{itemize}
	\end{uncoverenv}
	\footnotetext[1]{Seriously, never use \textit{typedef}.}
\end{frame}

\section{Exercise}
\subsection{}
\begin{frame}{Complex numbers}
	A \textit{complex number} $a + b\mathrm{i}$ can be handled as a composite of two real numbers (real part $a$ and imaginary part $b$).
	\begin{itemize}
		\item Write a program that is able to read and print \textit{complex number}s and stores them in an appropriate data structure
		\item The output should be in the form $a + b\mathrm{i}$, but you can ask to input $a$ and $b$ seperately
		\item \textbf{Experts:} write a function that takes a \textit{complex number} as an argument and returns its absolute value
		\begin{itemize}
			\item<2-> Hint: you have to pass a \textit{complex number}, but the return value is a real number
		\end{itemize}
	\end{itemize}
\end{frame}
\begin{frame}{Circles}
	A \textit{point} in 2D consists of two coordinates: x and y (both \textit{integer}). \\
	A \textit{circle} consists of a centre (\textit{point}), a radius (\textit{integer}), a circumference and an area (both \textit{float}).
	\begin{itemize}
		\item Write a program that reads two coordinates and a radius from the command line and stores them in the \textit{struct circle} described above
		\item Then the circumference and area are calculated and stored as well
		\item Afterwards the updated \textit{struct circle} is printed
		\item \textbf{Experts:} write a function that takes two \textit{circle}s as arguments and returns the \textit{circle} whose radius is larger
	\end{itemize}
\end{frame}

% nothing to do from here on
\end{document}
