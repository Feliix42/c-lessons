%% Nothing to modify here.
%% make sure to include this before anything else

\documentclass[10pt]{beamer}
\usetheme{Szeged}

% packages
\usepackage{color}
\usepackage{listings}

% color definitions
\definecolor{mygreen}{rgb}{0,0.6,0}
\definecolor{mygray}{rgb}{0.5,0.5,0.5}
\definecolor{mymauve}{rgb}{0.58,0,0.82}

% re-format the title frame page
\makeatletter
\def\supertitle#1{\gdef\@supertitle{#1}}%
\setbeamertemplate{title page}
{
  \vbox{}
  \vfill
  \begin{centering}
  \begin{beamercolorbox}[sep=8pt,center]{title}
      \usebeamerfont{supertitle}\@supertitle
   \end{beamercolorbox}
    \begin{beamercolorbox}[sep=8pt,center]{title}
      \usebeamerfont{title}\inserttitle\par%
      \ifx\insertsubtitle\@empty%
      \else%
        \vskip0.25em%
        {\usebeamerfont{subtitle}\usebeamercolor[fg]{subtitle}\insertsubtitle\par}%
      \fi%     
    \end{beamercolorbox}%
    \vskip1em\par
    \begin{beamercolorbox}[sep=8pt,center]{author}
      \usebeamerfont{author}\insertauthor
    \end{beamercolorbox}
    \begin{beamercolorbox}[sep=8pt,center]{institute}
      \usebeamerfont{institute}\insertinstitute
    \end{beamercolorbox}
    \begin{beamercolorbox}[sep=8pt,center]{date}
      \usebeamerfont{date}\insertdate
    \end{beamercolorbox}\vskip0.5em
    {\usebeamercolor[fg]{titlegraphic}\inserttitlegraphic\par}
  \end{centering}
  \vfill
}
\makeatother

% insert frame number
\expandafter\def\expandafter\insertshorttitle\expandafter{%
      \insertshorttitle\hfill%
\insertframenumber\,/\,\inserttotalframenumber}

% preset-listing options
\lstset{
  backgroundcolor=\color{white},   
  % choose the background color; 
  % you must add \usepackage{color} or \usepackage{xcolor}
  basicstyle=\footnotesize,        
  % the size of the fonts that are used for the code
  breakatwhitespace=false,         
  % sets if automatic breaks should only happen at whitespace
  breaklines=true,                 % sets automatic line breaking
  captionpos=b,                    % sets the caption-position to bottom
  commentstyle=\color{mygreen},    % comment style
  % deletekeywords={...},            
  % if you want to delete keywords from the given language
  extendedchars=true,              
  % lets you use non-ASCII characters; 
  % for 8-bits encodings only, does not work with UTF-8
  frame=single,                    % adds a frame around the code
  keepspaces=true,                 
  % keeps spaces in text, 
  % useful for keeping indentation of code 
  % (possibly needs columns=flexible)
  keywordstyle=\color{blue},       % keyword style
  % morekeywords={*,...},            
  % if you want to add more keywords to the set
  numbers=left,                    
  % where to put the line-numbers; possible values are (none, left, right)
  numbersep=5pt,                   
  % how far the line-numbers are from the code
  numberstyle=\tiny\color{mygray}, 
  % the style that is used for the line-numbers
  rulecolor=\color{black},         
  % if not set, the frame-color may be changed on line-breaks 
  % within not-black text (e.g. comments (green here))
  stepnumber=1,                    
  % the step between two line-numbers. 
  % If it's 1, each line will be numbered
  stringstyle=\color{mymauve},     % string literal style
  tabsize=4,                       % sets default tabsize to 4 spaces
  title=\lstname                   
  % show the filename of files included with \lstinputlisting; 
  % also try caption instead of title
}

% macro for code inclusion
\newcommand{\includecode}[2][c]{
	\lstinputlisting[caption=#2, style=custom#1]{#2}
}	% nothing to do here
%% Fill in metadata here that do not change over the course
%% They all are marked with the term "TODO". 
%% Search functions usually do the trick

% TODO select the targeted language
\usepackage[english]{babel}
% \usepackage[ngerman]{babel}

% TODO select the encoding
\usepackage[utf8]{inputenc}
% usepackage[latin1]{inputenc}

\newcommand{\course}{
	% TODO replace this with the name of the course
	cadvanced
}

\author{
	% TODO fill in the authors name
	Tony Template
}

\lstset{
	% TODO adapt these settings to your mainly used language
	% also see http://en.wikibooks.org/wiki/LaTeX/Source_Code_Listings
	% NOTE you can override these settings in individual cases 
	language = C,
	showspaces = false,
	showtabs = false,
	showstringspaces = false,
	escapechar = @
} % TODO modify this if you have not already done so

% meta-information
\newcommand{\topic}{
	File management
}

% nothing to do here
\title{\topic}
\supertitle{\course}
\date{}

% the actual document
\begin{document}

\maketitle

\begin{frame}{Contents}
	\tableofcontents
\end{frame}

\section{File descriptors}
\subsection{}

\begin{frame}[fragile]{Describing a file}
	The basic way to handle files in C is to use the interface with your operating system on a very low level.\\
	Each process has its own set of \textbf{file descriptors} - arbitrary integer numbers that represent a file.\\ \bigskip
	To open a file and get a descriptor use the system call function \textit{open()}:\\

	\begin{lstlisting}
int myfile = open("savestate", O_RDWR);
\end{lstlisting}

	\begin{itemize}
		\item See \textit{man 2 open} for further information and library includes concerning this function.
	\end{itemize}

\end{frame}

\begin{frame}[fragile]{Write and read}
	For actually putting data into a file, there is the function \textit{write()}.\\
	It takes a file descriptor, the input string and the amount of bytes to be written:\bigskip
	
	\begin{lstlisting}
int myfile = open("savestate", O_WRONLY | O_CREAT, 0644);
char *mytext = "foo";
write(myfile, mytext, 4);
\end{lstlisting}\ \\\ \\

	To read data from a file, you have to prepare a variable to put it in:\\\ \\
	
	\begin{lstlisting}
int myfile = open("savestate", O_RDONLY);
char mytext[256] = {0};
read(myfile, mytext, sizeof mytext - 1);
\end{lstlisting}
	
\end{frame}

\begin{frame}[fragile]{Close}
	After finishing your file operations, you have to tell your OS that you have do not need the file any more by closing the file descriptor:\\\ \\
	
	\begin{lstlisting}
int myfile = open("savestate", O_WRONLY | O_CREAT, 0644);
char *mytext = "foo";
write(myfile, mytext, 4);
close(myfile);
\end{lstlisting}\bigskip
	Note that there is a maximum number of files a process can have opened at a time!
\end{frame}

\begin{frame}[fragile]{Special file descriptors}
	By default, each process three file descriptors $-$ $0$, $1$ and $2$ $-$ referring to the standard input, output and error.\\
	You could read user input and write output by using the \textit{read()} and \textit{write()} functions along with these files as well:\bigskip
	\begin{lstlisting}
char message[32] = "Hello ";
write(1, "Enter your name:\n", 18);
read(0, message + 6, 24);
strcpy(message + strlen(message) - 1, "!\n");
write(1, message, strlen(message));
\end{lstlisting}\bigskip
	
	To avoid these random numbers in your code, use the constants defined in \textit{unistd.h}: \textit{STDIN\_FILENO}, \textit{STDOUT\_FILENO}, \textit{STDERR\_FILENO}.

\end{frame}

\section{File pointers}
\subsection{}

\begin{frame}[fragile]{Returning to standard}
	The C standard library contains a special datastructure for handling files: a so called \textbf{file pointer} \textit{FILE*}:\\\ \\
	
	\begin{lstlisting}
FILE *myfile = fopen("savestate", "r+");
char content[256] = {0};
fgets(content, sizeof content - 1, myfile);
fputs("foo", myfile);
fclose(myfile);
\end{lstlisting}\bigskip

	\begin{itemize}
		\item These and more handy functions are declared in \textit{stdio.h}
		\item See the man page for further information.
	\end{itemize}
\end{frame}

\begin{frame}[fragile]{Making life easier}
	By browsing \textit{stdio.h}, you will find the following function:
	\begin{lstlisting}
int fprintf(FILE *restrict, const char *restrict, ...);
\end{lstlisting}
	\textit{fprintf()} behaves exactly as \textit{printf()}, except that you can write to any file passed as the first argument.\\ \bigskip
	There are also special file pointers declared for standard input, output and error:
	\textit{stdin}, \textit{stdout} and \textit{stderr}.\\ \bigskip
	In fact, it is good practice to read user input with \textit{fgets()} since you can pass the size of your buffer avoiding dangerous \textbf{buffer overflows}:
	\begin{lstlisting}
fgets(buffer, sizeof buffer - 1, stdin);
\end{lstlisting}
\end{frame}

\section{Dungeon}
\subsection{}
\begin{frame}{Savestates}
	Now that you are common with file management, implement a savestate functionality into the dungeon.\\\ \\
	\begin{itemize}
		\item When the game quits (by pressing \textit{x}) the game data should be written into a savestate.
		\item If there is a savestate, the game should load it on start.
		\item There also should be an opportunity to exit the game without saving.
	\end{itemize}
\end{frame}

% nothing to do from here on
\end{document}
