
% den vorgegebenen header einbinden
%
% by Max Marx 2013, 
% edited by Fredo Erxleben 2014
%
\documentclass[12pt,a4paper]{scrartcl}
 
\usepackage[utf8]{inputenc}
\usepackage[ngerman]{babel}
%\usepackage{concrete}
%\usepackage{eulervm}
\usepackage{listings}
% \usepackage{etex}
\usepackage{tabularx}
\usepackage{color}

\definecolor{mygreen}{rgb}{0,0.6,0}
\definecolor{mygray}{rgb}{0.5,0.5,0.5}
\definecolor{mymauve}{rgb}{0.58,0,0.82}
 
\lstset{ %
  language=c,
  flexiblecolumns=false,
  basewidth={0.5em,0.45em},
  basicstyle=\ttfamily\upshape\small,
  sensitive=true,
  showstringspaces=false,
  numberstyle=\small,
  numberblanklines=true,
  showspaces=false,
  breaklines=true,
  showtabs=false
  tabsize=2,
  keywordstyle=\color{mymauve},
  stringstyle=\color{blue},
  numbers=left,
  numberstyle=\tiny\color{mygray},
  commentstyle=\color{mygreen}
}
\newcommand{\difficulty}{undefined}
\newcommand{\requirements}{none}
\newcommand{\aims}{learning Java}

\newcommand{\taskinfos}{
\renewcommand*{\arraystretch}{1.4}
\begin{center}
\begin{tabularx}{\textwidth}{|lX|}
    \hline
    \textbf{Difficulty:} & \difficulty \\
    \textbf{Requirements:} & \requirements \\
    \textbf{Aims:} & \aims \\
    \hline
\end{tabularx}
\end{center}
}

\date{} % empty because it should be ommited
\renewcommand*{\familydefault}{\sfdefault} 
% I expect the document to be read on a screen most of the time,
% so I am going for a sans-serif font

% ein paar Voreinstellungen,
% bitte entsprechend anpassen
\title{Feedback} % Name der Aufgabe
\author{} % Falls gewünscht, kann man sich als Author eintragen
\renewcommand{\difficulty}{Hard} % Schwierigkeitsgrad der Aufgabe
\renewcommand{\requirements}{Variables, input/output, conditions, ASCII} % Benötigte Voraussetzungen für die Aufgabe
\renewcommand{\aims}{Make decisions during run time, get to know ASCII more} % Lernziele der Aufgabe

\begin{document}
% Hier gibt es nichts zu tun
 \maketitle
 \taskinfos
% Ab hier kann man nach Belieben schalten und walten
% Teilaufgaben können mit sections, subsections und subsubsections abgetrennt werden.
% die Numerierung erfolgt automatisch

% \subsection{Ein Teilschritt}
% \subsubsection{Ganz genaue Anweisungen}
\ \\\ \\
\begin{itemize}
	\item Write a program, that asks the user to enter a character and answers whether the input is a letter, a number, or a special char.
		\item \textbf{Experts:} If the character is a small letter, also print the capital letter and vice versa.
\end{itemize}	
 
% und so geht Quelltext:
% Actung: Tabulatoren und Leerzeichen werden 1:1 übernommen
 
% Das sollte nicht fehlen:
\ \\\ \\\ \\\ \\\ \\
\section*{Hints}
	\begin{itemize}
		\item Have a look at the ASCII code table.
		\item \textbf{Experts:} Have a closer look at the ASCII code table
	\end{itemize}
 
\end{document}
